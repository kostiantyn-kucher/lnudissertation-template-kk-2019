\documentclass[notlud]{lnudissertation}
%\documentclass[notlud,showframe]{lnudissertation} % Display the printable area frames --- very useful when typesetting tables and figures

%\usepackage{graphicx}  % graphicx already loaded by the class
%% Pay attention to fontenc, inputenc, and babel
%\RequirePackage[]{hyphsubst}
\usepackage[T1]{fontenc}
\usepackage[utf8]{inputenc}
\usepackage[swedish,english]{babel}
\usepackage{mathptmx}
\usepackage{amsmath}
\usepackage{multirow}
\usepackage{enumerate}

%% Smarter management of references (e.g., to fix references like '[300, 15, 81, 1]')
\usepackage{cite}

 %% subfigure is deprecated, and subfig has issues with hyperref
\usepackage{subcaption}

%% booktabs are used for professionally looking tables: 
%% https://www.inf.ethz.ch/personal/markusp/teaching/guides/guide-tables.pdf
\usepackage{booktabs}

%% Used for sideways floats, which are probably necessary for this template, and rotation of custom stamps (see below)
\usepackage{rotating}

%% Used for custom stamps (see below)
\usepackage[absolute,overlay]{textpos}
\usepackage{pbox}



%% Font setup below; the order of loading and substituting the fonts is rather specific here

%% STEP 1 (preloading the sans serif font)
%% Biolinum from the libertine package is used as the sans serif font below: http://www.tug.dk/FontCatalogue/biolinum/
\usepackage{libertine}

%%% STEP 2 (optional)
%%% KK: libertine and substitutefonts are used to include several words with Cyrillic fonts in the acknowledgments (should work for Greek, too),
%%% so if you don't need that, you could remove them (but note that biolinum from the libertine package is used as the sans serif font below)
%\usepackage[scaled=.984]{libertinegc}
%\usepackage{substitutefont}

%% STEP 3 (the main font)
%% New PX (a Palatino clone) is used as the main font set for serif, typewriter, and math fonts: http://www.tug.dk/FontCatalogue/newpx/
\usepackage[largesc,looser,scaled=.92]{newpxtext}
\usepackage[scaled=.92]{newpxmath}
\linespread{1.05} % Give Palatino more leading (space between lines)

%% Other viable options for the main font set (supporting math mode) are, for example:
%% Charter BT http://www.tug.dk/FontCatalogue/charterbt/
%% Libertinus Serif http://www.tug.dk/FontCatalogue/libertinusserif/
%% EB Garamond http://www.tug.dk/FontCatalogue/ebgaramond/

%%% STEP 4 (optional)
%% See the note about substitutefonts above
%\substitutefont{T2A}{\rmdefault}{LinuxLibertineT-TLF}
%%% KK: Since I had to use a different font for Cyrillic, it looks wider, hence the usage of scalebox to squeeze it horizontally
%\newcommand{\cyrillicbox}[1]{\scalebox{0.88}[1.0]{\textcyrillic{#1}}}

%% STEP 5 (replacing the sans serif font)
%% biolinum will be used as the sans serif font in those few places where it is required
%% note: the libertine package is required for this!
\renewcommand{\sfdefault}{LinuxBiolinumT-TLF}

%% STEP 6
%% Microtype provides a number of adjustments for typesetting automatically, but if it's problematic for any reason, disable it
\usepackage{microtype}

%% End of font setup




%% KK: A funny command for a DRAFT stamp for first pages
%% Make sure to adjust the box size and position for textpos if page geometry or fonts are modified
\newcommand{\draftstamp}[1][DRAFT]{%
\begin{textblock*}{50mm}(100mm,25mm)%   % 100mm,25mm
\setlength{\fboxsep}{2mm}%
\rotatebox{15}{%
\fbox{\Large%
\textbf{#1}%
}%
}%
\end{textblock*}%
}

%% Several useful (or not so useful) commands
\newcommand{\cit}[1]{``#1''}
\newcommand{\ignore}[1]{}
\newcommand{\todo}[1]{\noindent\fcolorbox[rgb]{0.9,0,0}{1,1,0.8}{\parbox{0.97\columnwidth}{TODO: #1}}\newline}
\newcommand{\figbox}[1]{%
    {\colorlet{currentcolor}{.}%
    {\color{lightgray}%
	{\setlength{\fboxsep}{1pt}
    \fbox{\color{currentcolor}#1}}}}%
}

%% Additional hyphenation and word breaking settings below, if necessary:
%\hyphenation{}

%% It is recommended to start working in sloppy mode, and once the draft is complete, switch either to fussy or custom settings
%% and go over the text to avoid really wide inter-word gaps as well as overflows: 
%% https://tex.stackexchange.com/q/241343
%% http://open-juve.blogspot.com/2015/09/latex-sloppy-and-fussy-line-breaking.html
%% https://latex.org/forum/viewtopic.php?t=21170
\sloppy
%\fussy



%% kantlipsum is used purely to produce placeholder paragraphs with the \kant command for this example
%% Remove for actual documents
\usepackage{kantlipsum}


%% In case source code listings are used, uncomment the following:
%\usepackage{listings}


%% If there are some issues with some section marks in the headers at the first pages of the respective sections, 
%% see https://tex.stackexchange.com/a/94901


%% Setup the hyperlinks behavior + the metadata for the produced PDF file
\hypersetup{
	hidelinks=true, % Hide the boxes around hyperlinks
	pdfinfo={
		Title={Linn{\ae}us University Dissertation Example (non-LUD Series)},
		Author={John Doe},
		Subject={PhD Dissertation},
		Keywords={keyword1; keyword2},
  }
}

%----------------------------------------------------------------------------------------------------
%------------------------HERE STARTS THE PRINTING --------------------------------
%----------------------------------------------------------------------------------------------------
\begin{document}
%% Don't be surprised by *back*matter here, it is done to avoid page numbering, etc.
\backmatter

%% TODO: if necessary, generate a front cover page directly from Latex here
%% For relatively complex designs, it is probably a better idea to create cover pages *and the spine* in some external application

%% As of February 2019, LNU Press produce a separate leaflet ('spikblad') with a bibliographical reference and abstract to be inserted in the book, even for the books outside of LUD series (https://lnu.se/en/library/research-support/publish-with-lnu-press/checklist-notLUD/)
%% The front matter pages must be provided by the author in this case!
%% Traditionally, the halftitle page should get numbered as Roman 'i' (https://en.wikipedia.org/wiki/Book_design)
%% The order should be the following (you might need to modify this, if necessary...):
%% p. i: halftitle (a blank page with the title of the thesis)
%% p. iii: title page (series number, title, author, publisher)
%% p. iv: colophon (edition notice, ISBN, publisher, printer)
%% p. v:  dedication
%% p. vii: abstract
%% p. ix: acknowledgments
%% p. xi: table of contents, and so on
%% p. i: half-title
\begin{center}
\null\vspace{4.5cm}
{ \Large \bfseries Linn{\ae}us University Dissertation Example\\[0.3cm](non-LUD Series)}%\\
\end{center}

\draftstamp
%\draftstamp[PREPRINT]

\cleardoublepage

%% p. iii: title page
\begin{center}

%% Upper part of the page
\null\vspace{4.5cm}
\normalsize   John Doe\\[1.5cm]
{ \Large \bfseries Linn{\ae}us University Dissertation Example\\[0.3cm](non-LUD Series)}%\\
%\large 
\\[2cm]

Doctoral Dissertation\\[0.2cm]
Subject/Discipline\\[0.4cm]
20XX
\vfill

\draftstamp
%\draftstamp[PREPRINT]

%% Bottom of the page
%% The LNU logo image is copied from
%% https://medarbetare.lnu.se/medarbetare/stod-och-service/kommunikation-och-marknadsforing/designmanual/grundelement/logotyp/logotyp--symbol/
\includegraphics[scale=0.2]{images/lnu.pdf}%

\end{center}

%% Note: the following page should appear in verso, therefore don't use clear*double*page below
\clearpage

%% p. iv: colophon / edition notice
\null
\vfill
\begin{flushleft} 
{\small PhD Dissertation Presented in Fulfillment of the Requirements for the Degree of Doctor of Philosophy in ??? at Linnaeus University, Month Day, 20XX.\\[0.6cm]
\textbf{Linn{\ae}us University Dissertation Example (non-LUD Series)}
\\[0.6cm]
John Doe\\[0.6cm]
Linnaeus University\\[0.1cm]
Department of ???\\[0.1cm]
SE-351 95 V\"axj\"o, Sweden, 20XX\\[0.1cm]
https://lnu.se/\\[0.6cm]
ISBN: 978-91-?????-??-? (print), 978-91-?????-??-? (pdf)\\[0.1cm]
Published by: ???, 351 95 V\"axj\"o\\[0.1cm]
Printed by: ???, 20XX}
\end{flushleft}

\draftstamp
%\draftstamp[PREPRINT]

\cleardoublepage

\null
\vfill
\noindent
\begin{minipage}{\textwidth}%
\raggedleft%
\emph{Dedicated to\ldots}%
\end{minipage}%
\vfill

\cleardoublepage
\frontmatter
\begin{abstract}
% Abstract here
% 300--350 words (https://lnu.se/en/library/forskningsstod/publish-with-lnu-press/checklistLUD/)
\kant[1-3]

%% With the null, we avoid the space when starting a new paragraph, if necessary
%\null
\bigskip
\vfill
\noindent
\textbf{Keywords:}
keyword1, keyword2

\end{abstract}

\cleardoublepage

\begin{acknowledgments}
\kant[4-5]

%% With the null, we avoid the space when starting a new paragraph, if necessary
%\null
%\bigskip
%\vfill
%\noindent V\"axj\"o, Sweden\\
%\noindent 2019\\
%\bigskip

\end{acknowledgments}
\cleardoublepage



%% Start the ToC and add the corresponding lists
\renewcommand\contentsname{Table of Contents}
\setcounter{tocdepth}{1}

\phantomsection
\addcontentsline{toc}{chapter}{\contentsname{}}
%% To squeeze one more line of contents in the ToC page, if neccessary, modify the length as below:
%\addtolength{\cftaftertoctitleskip}{-11pt}
\tableofcontents
\cleardoublepage

%% With regard to figure and table sizes, the following could be used (figbox simply provides a gray frame):
%% wide figure: \figbox{\includegraphics[width=0.975\linewidth]{images/zzz.pdf}}
%% narrow figure: \figbox{\includegraphics[width=0.66\linewidth]{images/zzz.pdf}} (well, might be different than 0.66 depending on the context)
%% sideways figure: \begin{sidewaysfigure} ... \figbox{\includegraphics[width=0.985\linewidth]{images/zzz.pdf}} (make sure to check the margins with geometry package option 'showframe' !)
%% subfigures: \begin{subfigure}{\linewidth} ... \figbox{\includegraphics[width=0.975\linewidth]{images/zzz.pdf}} (make sure configure space between subfigures and captions properly)
%% tables: if required, \begin{table}[t] ... \begin{minipage}{0.975\textwidth} ... \begin{tabular}{lll}
%% Also, note that percent signs at the end of some commands/lines might be important to avoid extra space

\phantomsection
\addcontentsline{toc}{chapter}{\listfigurename{}}
\listoffigures
\cleardoublepage

\phantomsection
\addcontentsline{toc}{chapter}{\listtablename{}}
\listoftables
\cleardoublepage

%% The list of listings is not tested!!
%% Add the listings package in case it is needed, too
%\lstlistoflistings

% Add the list of publications
\chapter*{List of Publications}
\addcontentsline{toc}{chapter}{List of Publications}
\markboth{List of Publications}{List of Publications}

\textbf{This dissertation is based on the following refereed publications in chronological order} 
(I have contributed to all stages of work as the lead author):
%% To prepare the entries for this list, add references to your papers like "\cite{Myself2019a}\cite{Myself2019b}" here, 
%% generate the PDF with Latex, then open the generated BBL file with a text editor, 
%% copy the necessary entries here, and finally, edit them with extra notes about Chapters, etc.
\begin{enumerate}
\item Myself and My Coauthor. 
Paper title: Paper subtitle. 
In {\em Proceedings of The Best Conference}, short paper track, BestConf~'15, pages 100--105. Publisher, 2015. 
\href{https://doi.org/xxx}{\path{doi:xxx}}. 
Materials appear in Chapter~X.

\item Myself, My Coauthor, and Another Coauthor. 
My journal article. 
{\em Journal Title}, 10(1):100--115, April 2016. 
\href{https://doi.org/yyy}{\path{doi:yyy}}. 
Materials appear in Chapter~X.

\item \ldots
\end{enumerate}

\noindent Additionally, the materials of Chapter~X have been used to prepare the following full paper manuscript:
\begin{itemize}
\item Myself and My Coauthor. 
Manuscript title: Manuscript subtitle.
2019. 
\end{itemize}

%\bigskip
\newpage

\noindent\textbf{Further publications not related to this dissertation}
(I have contributed to all or some stages of work in conceptualization, implementation, or writing):
\begin{enumerate}
\item Myself, My Coauthor, and Another Coauthor. 
A not so relevant paper title. 
In {\em Proceedings of Another Conference}, Conf~'16, pages 20--30. Publisher, 2016. 
\href{https://doi.org/zzz}{\path{doi:zzz}}. 
\end{enumerate}

\cleardoublepage

%----------------------------------------------------------------------------------------------------
%---------------------HERE STARTS THE THESIS CONTENT-------------------------
%----------------------------------------------------------------------------------------------------
\mainmatter

%% Input chapters here
\chapter{Introduction}\label{ch:introduction}

\chaptertoc

\noindent \kant[7-8]

Also, use references to entries of several types here: 
\begin{itemize}
\item a book~\cite{Card1999};
\item a book chapter with separate authors~\cite{Fekete2008};
\item a journal article~\cite{VanWijk2006a};
\item a conference paper~\cite{Shneiderman1996};
\item a URL link~\cite{ColorBrewer}; and
\item several references~\cite{Shneiderman1996,Card1999,VanWijk2006a,Fekete2008,ColorBrewer} used in arbitrary order~\cite{VanWijk2006a,ColorBrewer} to check if automatic sorting with \cit{cite} works.
\end{itemize}

\noindent Test a footnote here, too.\footnote{A footnote}. 
Test the quotation environment with a URL link:
\begin{quotation}
\centering
\url{https://lnu.se/}
\end{quotation}

\noindent Test a quotation environment with a text quote:
\begin{quote}
\emph{\cit{A very smart and deep quote \dots}}
\end{quote} 
\noindent Also test another footnote with a URL link.\footnote{\url{https://lnu.se/} (last accessed in February 2019)}


%% The way the section title and mark are defined is to force the correct section mark to appear at the correct page
%% See more details at https://tex.stackexchange.com/a/94901
%% This might not be necessary for all the cases, though
\section[Motivation for Our Problem]{Motivation for Our Research Problem%
\sectionmark{Motivation}%
}\label{sec:intro-motivation}
\sectionmark{Motivation}

\begin{figure}[t!]
\centering
	%\figbox{\includegraphics[width=0.975\linewidth]{images/zzz.pdf}}%
    \figbox{\rule{.1pt}{2cm} \rule[1cm]{2cm}{.1pt} \rule{.1pt}{2cm}}
	\caption[Short caption for example figure]{Long figure caption.
	Explain the contents of the figure here properly.}%
	\label{fig:introduction-example}%
\end{figure} 

\kant[9-15]


\begin{sidewaysfigure}
\centering
	%% One might have to decrease the size of the sideways figure to fit the page margins
	%\figbox{\includegraphics[width=0.93\linewidth]{images/zzz.png}}%
    \figbox{\rule{.1pt}{5cm} \rule[2.5cm]{13cm}{.1pt} \rule{.1pt}{5cm}}
	%% ... and to shrink space to fit the page margins
	%\vspace{-2mm}
	\caption[Sideways figure example]{A long caption for the sideways figure here.}%
	\label{fig:introduction-example-sideways}%
	%\vspace{-2mm}
\end{sidewaysfigure}

\begin{figure}[t!]
%\centering
	\begin{subfigure}[t]{0.477\linewidth}
			\centering
			%\figbox{\includegraphics[width=\linewidth]{images/zzz.png}}%
			\figbox{\rule{.1pt}{2cm} \rule[1cm]{2cm}{.1pt} \rule{.1pt}{2cm}}
			\caption{}%
	\end{subfigure}%
	\hspace{3pt}
	\begin{subfigure}[t]{0.4754\linewidth}
			\centering
			%\figbox{\includegraphics[width=\linewidth]{images/zzz.png}}%
			\figbox{\rule{.1pt}{2cm} \rule[1cm]{2cm}{.1pt} \rule{.1pt}{2cm}}
			\caption{}%
	\end{subfigure}%
	%\vspace{-2mm}
	\caption[Short caption for a figure with subfigures]{A figure with subfigures with long captions. 
	(a) A dedicated caption could be provided directly in the subfigure code, but a long caption text arguably suits this area here better. 
    (b) The same applies to the second subfigure.
	}%
	\label{fig:intro-subfigures}%
\end{figure} 


\subsection{Subsection Here}\label{subsec:intro-subsection}

\begin{table}[t!]
\caption[Short caption for a table]{An arbitrary table}%
\label{tab:introduction-table}%
%\renewcommand{\arraystretch}{1.2}
\begin{minipage}{0.975\textwidth}
\centering
\begin{tabular}{lll}
%% The header
\toprule 
\parbox{0.26\textwidth}{\centering\textbf{Title}}
&
\parbox{0.32\textwidth}{\centering\textbf{Description}}
&
\parbox{0.33\textwidth}{\centering\textbf{Examples}} \\ 
\midrule
%% The body
\parbox[c][9mm]{0.26\textwidth}{Foo, bar,\\[-2pt]and baz}
&
\parbox[c]{0.32\textwidth}{ %
\scriptsize
Expression of foo and bar}
&
\parbox[c]{0.33\textwidth}{ %
\vspace{2pt}
\scriptsize
\emph{Example1}; \emph{Example2}
\vspace{2pt}} \\
\parbox[c][9mm]{0.26\textwidth}{Foo, bar,\\[-2pt]and baz}
&
\parbox[c]{0.32\textwidth}{ %
\scriptsize
Expression of foo and bar}
&
\parbox[c]{0.33\textwidth}{ %
\vspace{2pt}
\scriptsize
\emph{Example1}; \emph{Example2}
\vspace{2pt}} \\
%% The footer
\bottomrule
\end{tabular}% 
\end{minipage}%

\bigskip
\raggedright
\footnotesize{\emph{Note:} Adjust the column widths appropriately. 
And this is the area for long table caption notes, by the way. 
}
\end{table} 


\kant[28-31]


\begin{sidewaystable}
%\vspace{-4mm}
\caption[Short caption for a complex sideways table]{A complex sideways table consisting of several parts}
\label{tab:introduction-sideways-table}
\centering
\setlength{\tabcolsep}{2pt}
\renewcommand{\arraystretch}{1.2}
\setlength\doublerulesep{2mm} 
\footnotesize
\begin{minipage}[t]{0.27\textwidth}
\begin{tabular}[t]{lr}
\toprule
\textbf{Group header} & \textbf{100}\\ 
\midrule
Foo & 75\\ 
Bar & 20\\
Baz & 5\\
\addlinespace
\addlinespace
\textbf{Group header} & \textbf{100}\\ 
\midrule
Foo & 75\\ 
Bar & 20\\
Baz & 5\\
\addlinespace
\addlinespace
\textbf{Group header} & \textbf{100}\\ 
\midrule
Foo & 75\\ 
Bar & 20\\
Baz & 5\\
\bottomrule
\end{tabular} 
\end{minipage}
\hspace{1mm}
\begin{minipage}[t]{0.445\textwidth}
\begin{tabular}[t]{lr}
\toprule
\textbf{Group header} & \textbf{100}\\ 
\midrule
Foooooooooooooooooo & 75\\ 
Barrrrrrrrrrrrrrrrrrrrrrrrrrr & 20\\
Bazzzzzzzzzzzzzzzzzzz & 5\\
\addlinespace
\addlinespace
\textbf{Group header} & \textbf{100}\\ 
\midrule
Foooooooooooooooooo & 75\\ 
Barrrrrrrrrrrrrrrrrrrrrrrrrrr & 20\\
Bazzzzzzzzzzzzzzzzzzz & 5\\
\bottomrule
\end{tabular} 
\end{minipage}
\hspace{1mm}
\begin{minipage}[t]{0.22\textwidth}
\begin{tabular}[t]{lr}
\toprule
\textbf{Group header} & \textbf{100}\\ 
\midrule
Foo & 75\\ 
Bar & 20\\
Baz & 5\\
\addlinespace
\addlinespace
\textbf{Group header} & \textbf{100}\\ 
\midrule
Foo & 75\\ 
Bar & 20\\
Baz & 5\\
\addlinespace
\addlinespace
\textbf{Group header} & \textbf{100}\\ 
\midrule
Foo & 75\\ 
Bar & 20\\
Baz & 5\\
\bottomrule
\end{tabular} 
\end{minipage}
%\vspace{-5mm}

\bigskip
\raggedright
\footnotesize{\emph{Note:} Adjust the minipage widths appropriately.
}
\end{sidewaystable}

\subsubsection{Subsubsection Here}\label{subsubsec:intro-subsubsection}
\kant[32]

\paragraph{A Named Paragraph}
\kant[33]




\cleardoublepage
\chapter{Background}\label{ch:background}

\chaptertoc

\noindent \kant[16]

\section{Underlying Problems in Another Disciplines}\label{sec:background-underlying}

\kant[17-20]

\begin{figure}[t!]
\centering
	%\figbox{\includegraphics[width=0.975\linewidth]{images/zzz.pdf}}%
    \figbox{\rule{.1pt}{2cm} \rule[1cm]{2cm}{.1pt} \rule{.1pt}{2cm}}
	\caption[Note the gap above]{Note the gap between the chapters in the lists of figures and tables.}%
	\label{fig:background-example}%
\end{figure} 

\cleardoublepage
\chapter[Related Work]{Related Work and Design Space for Our Research Problem}\label{ch:related-work}
\chaptermark{Related Work and Design Space}

\chaptertoc

\noindent \kant[21]

\section{The First Group of Related Approaches}\label{sec:related-work-first-group}

\kant[22-26]

\cleardoublepage
%% ...
\chapter[Conclusions and Future Work]{Conclusions and Future Work}\label{ch:conclusions}

\chaptertoc

\noindent \kant[27]

\section{Research Findings}\label{sec:conclusions-findings}

First of all, \ldots



\cleardoublepage

%----------------------------------------------------------------------------------------------------
%---------------------HERE STARTS THE APPENDIX -----------------------------------
%----------------------------------------------------------------------------------------------------

%\addcontentsline{toc}{chapter}{Appendix}
%\appendix
%\input{appendix01.tex}

\phantomsection
\addcontentsline{toc}{chapter}{Bibliography}

\bibliographystyle{plainurl}
{
\footnotesize 
%\bibliography{bibliography-own,bibliography-main,bibliography-additional}
\bibliography{example-bibliography}
}
\backmatter
\cleardoublepage

%% TODO: if necessary, generate a back cover page directly from Latex here

\end{document}
